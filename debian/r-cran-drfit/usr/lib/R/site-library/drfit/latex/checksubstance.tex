\HeaderA{checksubstance}{Check raw data for a specified substance}{checksubstance}
\keyword{database}{checksubstance}
\begin{Description}\relax
Report metadata for a specified substance from a specified database, and plot
the data.
\end{Description}
\begin{Usage}
\begin{verbatim}
  checksubstance(substance,db="cytotox",experimentator="%",celltype="%",enzymetype="%",whereClause="1",ok="%")
\end{verbatim}
\end{Usage}
\begin{Arguments}
\begin{ldescription}
\item[\code{substance}] The name of the substance identifying it within the database.
\item[\code{db}] The database to be used. Currently, the databases "cytotox" and "enzymes"
of the UFT Department of Bioorganic Chemistry are supported (default is
"cytotox").
\item[\code{experimentator}] The name of the experimentator whose data is to be used. Default is "\%", which
means that data from all experimentators are shown.
\item[\code{celltype}] Currently, only data for IPC-81, C6, NB4, HeLa, Jurkat and U937 are supported.
Default is "\%", i.e. data for any cell type will be displayed.
\item[\code{enzymetype}] Currently, only data for AChE, GR and GST are supported. The default value is "\%", i.e. data for any enzyme type will be displayed.
\item[\code{whereClause}] With this argument, additional conditions for the SQL query can be set, 
e.g. "where plate != 710". The default is 1 (in SQL syntax this means TRUE).
\item[\code{ok}] With the default value "\%", all data in the database is retrieved for the 
specified substance.
\end{ldescription}
\end{Arguments}
\begin{Value}
The function lists a report and shows one graph.
\end{Value}
\begin{Author}\relax
Johannes Ranke 
\email{jranke@uni-bremen.de} 
\url{http://www.uft.uni-bremen.de/chemie/ranke}
\end{Author}
\begin{Examples}
\begin{ExampleCode}
# Check substance IM14 BF4 in the cytotox database
## Not run: checksubstance("IM14 BF4")
\end{ExampleCode}
\end{Examples}

