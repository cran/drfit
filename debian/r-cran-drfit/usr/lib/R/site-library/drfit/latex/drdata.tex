\HeaderA{drdata}{Get dose-response data}{drdata}
\keyword{IO}{drdata}
\keyword{database}{drdata}
\begin{Description}\relax
Get dose-response data from a remote mysql server
\end{Description}
\begin{Usage}
\begin{verbatim}
  drdata(substances, experimentator = "%", db = "cytotox", celltype = "IPC-81", 
    enzymetype="AChE", whereClause = "1", ok = "'ok','no fit'")
\end{verbatim}
\end{Usage}
\begin{Arguments}
\begin{ldescription}
\item[\code{substances}] A string or an array of strings with the substance names for
which dose-response data is to be retrieved.
\item[\code{experimentator}] The name of the experimentator whose data is to be used. Default is "
means that data from all experimentators are retrieved.
\item[\code{db}] The database to be used. Currently, the databases "cytotox" and "enzymes"
of the UFT Department of Bioorganic Chemistry are supported (default is
"cytotox").
\item[\code{celltype}] Currently, only data for IPC-81, C6, NB4, HeLa, Jurkat and U937 are supported.
\item[\code{enzymetype}] Currently, only data for AChE, GR and GST are supported.
\item[\code{whereClause}] With this argument, additional conditions for the SQL query can be set, 
e.g. "where plate != 710". The default is 1 (in SQL syntax this means TRUE).
\item[\code{ok}] With the default value "'ok','no fit'", only data that has been checked and set to "ok"
or "no fit" in the database is retrieved. The argument "no fit" will result
in not using the data for fitting, but it will be plotted.
Another sensible argument would be "'ok','no fit','?'", in order to additionally
retrieve data which has not yet been checked.
\end{ldescription}
\end{Arguments}
\begin{Details}\relax
The function is currently only used for retrieving data from the
mysql database "cytotox" of the UFT Department of Bioorganic Chemistry.
Access to this database is limited to UFT staff.  Additionally to the
installation of the RODBC package, it is required to set up a ODBC data
source with the name "cytotox", using an ODBC driver for mysql, probably
myODBC. Then, under Unix, you can use iodbc or unixodbc for setting up the
respective data source with data source name (DSN) "cytotox". For my
setting using unixodbc, I am using the file \file{/etc/odbcinst.ini}
containing: 
\Tabular{lll}{
[MySQL] & & \\
Description & = & MySQL driver for ODBC \\
Driver & = & /usr/local/lib/libmyodbc.so \\
Setup & = & /usr/lib/odbc/libodbcmyS.so \\
}
and the file \file{/etc/odbc.ini} containing:
\Tabular{lll}{
[cytotox] & & \\
Description & = & Cytotoxicity database of the department of bioorganic chemistry, UFT Bremen \\
Driver & = & MySQL \\
Trace & = & Yes  \\
TraceFile & = & /tmp/odbc.log  \\
Database & = & cytotox  \\
Server & = & eckehaat  \\
Port & = & 3306  \\
}.
\end{Details}
\begin{Value}
\begin{ldescription}
\item[\code{data}] A data frame with a factor describing the dose levels, the numeric dose levels
and a numeric column describing the response, as well as the entries for
plate, experimentator, performed (date of test performance), celltype, unit
(of the dose/concentration), and for the ok field in the database.
\end{ldescription}
\end{Value}
\begin{Author}\relax
Johannes Ranke 
\email{jranke@uni-bremen.de} 
\url{http://www.uft.uni-bremen.de/chemie/ranke}
\end{Author}
\begin{Examples}
\begin{ExampleCode}
# Get cytotoxicity data for Tributyltin and zinc pyrithione, tested with IPC-81 cells
## Not run: data <- drdata(c("TBT","ZnPT2"))
\end{ExampleCode}
\end{Examples}

