\HeaderA{linlogitf}{Linear-logistic function}{linlogitf}
\keyword{models}{linlogitf}
\keyword{regression}{linlogitf}
\keyword{nonlinear}{linlogitf}
\begin{Description}\relax
Helper function describing a special type of dose-response curves, showing a stimulus
at subtoxic doses.
\end{Description}
\begin{Usage}
\begin{verbatim}
  linlogitf(x,k,f,mu,b)
\end{verbatim}
\end{Usage}
\begin{Arguments}
\begin{ldescription}
\item[\code{x}] In this context, the x variable is the dose.
\item[\code{k}] In the drfit functions, k is set to 1.
\item[\code{f}] One of the parameters describing the curve shape.
\item[\code{mu}] The parameter describing the location of the curve (log EC50).
\item[\code{b}] One of the parameters describing the curve shape.
\end{ldescription}
\end{Arguments}
\begin{Value}
The response at dose x.
\end{Value}
\begin{Author}\relax
Johannes Ranke 
\email{jranke@uni-bremen.de} 
\url{http://www.uft.uni-bremen.de/chemie/ranke}
\end{Author}
\begin{References}\relax
van Ewijk, P. H. and Hoekstra, J. A. (1993) \emph{Ecotox Environ Safety}
\bold{25} 25-32
\end{References}

